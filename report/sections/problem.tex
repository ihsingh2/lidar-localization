\section{Problem Statement}
\label{sec:problem_statement}

Global localization using LiDAR (Light Detection and Ranging) has emerged as a pivotal technology in autonomous robotics, enabling precise positioning within a pre-existing map without relying on initial pose estimates. This capability is especially crucial in environments where GPS signals are unreliable or unavailable, such as urban canyons, dense forests, or indoor settings. Traditional global localization methods often grapple with the challenges of large search spaces and computational inefficiencies, particularly in 3D environments.

The Branch-and-Bound (BnB) algorithm has been a cornerstone in optimization problems, offering a systematic approach to traverse and prune the search space efficiently. The 2D variant, known as BBS (Branch-and-Bound Scan Matching), has demonstrated success in real-time loop closure detection within SLAM (Simultaneous Localization and Mapping) systems \cite{Hess2016}. Extending BBS to 3D, however, introduces significant challenges in memory consumption and computational overhead due to the exponential increase in the number of possible poses.

This paper examines a 3D global localization framework, termed 3D-BBS \cite{aoki20243dbbsgloballocalization3d}, which integrates the BnB algorithm with efficient voxel mapping and GPU-accelerated computations. We employ KISS-ICP \cite{Vizzo_2023} as a prerequisite for generating accurate 3D LiDAR maps. Utilizing the KITTI Odometry dataset \cite{Geiger2013}, we conduct ablation studies on key configuration parameters of the 3D-BBS algorithm. Additionally, we explore the integration of RGB images from the KITTI Odometry dataset to generate local point clouds, using metric depth estimates computed using Depth Anything v2 \cite{yang2024depthv2}.

The remainder of this paper is organized as follows: Section \ref{sec:accomplishment} summarizes the progress made in the project. Section \ref{sec:literature_review} reviews related literature in global localization and BnB-based point cloud matching. Section \ref{sec:dataset} briefly describes the dataset. Section \ref{sec:methodology} details the proposed 3D-BBS methodology, including algorithmic enhancements and mathematical formulations. Section \ref{sec:results} presents the experimental setup, ablation studies, and visualization. Section \ref{sec:contributions} lists the individual contributions to the project. Finally, Section \ref{sec:conclusion} concludes the paper and outlines future directions.
